% Author: Kurt O'Hearn
% Author: Nicholas Olesak
% Date: April 15, 2013
% Class: CIS 343-02
% Purpose: Progamming Language Survey Term Paper on Haskell
\documentclass[titlepage,12pt]{article}

\usepackage{minted}

% aliases
\newcommand{\be}{\begin{enumerate}}
\newcommand{\bi}{\begin{itemize}}
\newcommand{\ee}{\end{enumerate}}
\newcommand{\ei}{\end{itemize}}

\title{Programming Language Survey Paper on Haskell}
\author{
        Kurt O'Hearn\\
        Nicholas Olesak\\
}
\date{\today}


\begin{document}

\maketitle 
\setcounter{tocdepth}{3}
\tableofcontents \newpage

\section{Introduction}
\bi
    \item Origin and history
    \item Explanation of the functional programming paradigm
    \item Application domains of the language
\ei
[TODO]~\cite{learnyouahaskell}
\begin{minted}{haskell}
module Main where
import Data.List

main :: IO ()
main = putStrLn "Hello World!" >> test

test = do 
    print 123
\end{minted}

\section{Data and Control Abstractions}


\subsection{Data Abstractions}
\subsubsection{Data Types}
\bi
    \item Basis data types: Int, Float, Char, Boolean, lists
\begin{minted}{haskell}
let i = 2
let x = 3.14
let c = 'c'
let str = ['H','a','s','k','e','l','l','!']
; shorthand for the above
let str2 = "Haskell!"
\end{minted}

    \item Type variables: variables that can be any type (used in polymorphic functions)
\begin{minted}{haskell}
lastTwo :: [a] -> [a]
lastTwo a = [a !! ((length a) - 2), a !! ((length a) - 1)]
\end{minted}
\ei

\subsubsection{Type Checking}
\bi
    \item Haskell: static typing, type inferencing by context
\begin{minted}{haskell}
let myStr = "2"
; will result in an exception as the desired type to convert to is
; ambiguous since Haskell has no context to infer the type
let myNum = read myStr
; not ambiguous, as the compiler infers the Int type is desired
let myNum2 = (read myStr) + 2
\end{minted}
    \item Type classes: Eq, Ord, Show/Read, Bounded, Enum, Num/Integral/Floating
\ei

\subsection{Control Abstractions}
\subsubsection{Expressions}
\bi
    \item Function definitions
    \item Binding: let, where
\ei

\subsubsection{Operators and Precedence}
\bi
    \item Operators: postfix notation, naming typically consists of symbols
    \item Functions: infix notation, named using alphanumeric characters in camel case
    \item Any function can be used as an operator by calling it in postfix notation
\begin{minted}{haskell}
f (Int a) => a -> a
f a b = a + b
; invoke the function in postfix notation
print (1 `f` 2)
\end{minted}
    \item Any operator can be called as a function in infix notation
\begin{minted}{haskell}
print ((+) 2 5)
\end{minted}
    \item Operators can be overloaded
\begin{minted}{haskell}
* a b = a * a * b
print (2 * 3)
\end{minted}
\ei

\subsubsection{Selection Constructs}
\bi
    \item Pattern matching
    \item if/then/else
    \item Guards
    \item Case expression
\ei

\subsubsection{Iterative Constructs}
\bi
    \item List comprehensions
    \item Map/fold
    \item Recursion
\ei

\subsubsection{Functions}
\bi
    \item Definition/use
    \item Support parameter passing techniques
\ei

\subsubsection{Scoping}
\bi
    \item Static
\ei

\subsubsection{Modules}

\subsubsection{Exception Handling}


\section{Advanced Topics}

\subsection{Inheritance}
\subsection{Concurrency Support}
\subsection{Introspection}

\bibliographystyle{abbrv}
\bibliography{paper}

\end{document}
